\documentclass[11pt]{article}
\usepackage{latexsym}
\usepackage{amsmath,amssymb,amsthm}
\usepackage{epsfig}
\usepackage[right=0.8in, top=1in, bottom=1.2in, left=0.8in]{geometry}
\usepackage[colorlinks=true,urlcolor=Blue,citecolor=Blue,linkcolor=BrickRed]{hyperref}
\usepackage[dvipsnames,usenames]{xcolor}

\usepackage{setspace}
\usepackage[utf8]{inputenc}

%\spacing{1.06}

\newcommand{\handout}[5]{
  \noindent
  \begin{center}
  \framebox{
    \vbox{\vspace{0.25cm}
      \hbox to 5.78in { {University of Wrocław:\hspace{0.12cm}Algorithms for
          Big Data (Fall'19)} \hfill #2 }
      \vspace{0.48cm}
      \hbox to 5.78in { {\Large \hfill #5  \hfill} }
      \vspace{0.42cm}
      \hbox to 5.78in { {#3 \hfill #4} }\vspace{0.25cm}
    }
  }
  \end{center}
  \vspace*{4mm}
}

\newcommand{\lecture}[4]{\handout{#1}{#2}{#3}{Scribe:\hspace{0.08cm}#4}{Lecture #1}}

\newtheorem{theorem}{Theorem}
\newtheorem{corollary}[theorem]{Corollary}
\newtheorem{lemma}[theorem]{Lemma}
\newtheorem{observation}[theorem]{Observation}
\newtheorem{example}[theorem]{Example}
\newtheorem{definition}[theorem]{Definition}
\newtheorem{claim}[theorem]{Claim}
\newtheorem{fact}[theorem]{Fact}
\newtheorem{assumption}[theorem]{Assumption}

\newcommand{\E}{{\mathbb E}}
\DeclareMathOperator{\var}{Var}
\newcommand{\eps}{\varepsilon}
\newcommand{\bigo}{\mathcal{O}}
\setcounter{MaxMatrixCols}{20}

\begin{document}

\lecture{9: Sparse Recovery}{ 09/12/2019}{Lecturer: \emph{Przemysław Uznański }}{-}


\section{Sparse FFT - fast for random signal}

Assume we have fourier-sparse signal, where frequencies were sampled uniformly at random, that is $\hat{a}$ support is $\{u_1,\ldots,u_k\}$ where each $u_i$ was picked independently.

We use following:
\begin{theorem}[Aliasing Theorem]
Given signal $a = (a_0,\ldots,a_{n-1})$, let $L$ divide $n$, and consider $b = (a_0,a_{L},a_{2L},\ldots,a_{n-L})$. Then $\hat{b}_i = \frac{1}{L}\sum_{l = 0}^{L-1} \hat{a}_{i+l \cdot n/L}$.
\end{theorem}
\begin{proof}
\begin{align*}
 \sum_{l = 0}^{L-1} \hat{a}_{i+l \cdot n/L} &= \sum_{l = 0}^{L-1} \sum_{j=0}^{n} a_j \omega^{(i+l \cdot n/L)j}\\
 &= \sum_{j=0}^{n}  a_j \omega^{ij}  \sum_{l = 0}^{L-1} \omega^{l j \cdot n/L}\\
 &= \sum_{j=0}^{n}  a_j \omega^{ij}  \sum_{l = 0}^{L-1} (\omega^{j \cdot n/L})^l\\
 &= \sum_{j=0}^n a_j \omega^{ij} \cdot L \cdot [j = 0 \bmod L]\\
 &= L \sum_{j=0}^{n/L-1} a_{jL} \omega^{ijL} \\
 &= L \sum_{j=0}^{n/L-1} b_j (\omega^{iL})^j \\
 &= L \hat{b}_i
\end{align*}

\end{proof}

We thus use a following algorithm:

\begin{itemize}
\item $L = \bigo(n/k^2)$.
\item Let $a' = (a_0,a_{L},\ldots,a_{n-L})$ of length $\bigo(k^2)$.
\item Let $a'' = (a_1,a_{L+1},\ldots,a_{n-L+1})$  of length $\bigo(k^2)$.
\item Compute $\hat{a}'$ and $\hat{a}''$.
\item Iterate through non-zero elements of $\hat{a}'$, say $\hat{a}'_u$ and apply two-point algorithm for $\hat{a}'_u$ and $\hat{a}''_u$.
\end{itemize}

Why does it work (and what is two-point algorithm)?
First, $\hat{a}'_u = \frac1L\sum_{l=0}^{L-1} \hat{a}_{u + l n/L}$. On the other hand, by using timeshift-frequencyshift theorem, $a''$ is a sampling from shifted-by-one $a$, so $\hat{a}''_u = \frac1L\sum_{l=0}^{L-1} \hat{a}_{u+ln/L} \omega^{u+ln/L}$.


Any non-zero element of $\hat{a}$ falls (with $2/3$ ppb) into distinct ``buckets'' (birthday paradox), so $\hat{a}'_u = \frac1L\hat{a}_{u+ in/L}$ for some unknown $l$, and $\hat{a}''_u = \frac1L\hat{a}_{u+in/L} \omega^{u+in/L}$. Thus
$\hat{a}''_u / \hat{a}'_u = \omega^{u+in/L}$ which reveals $i$.

Thus total runtime is $\bigo(k^2 \log k)$ for $2/3$ success probability.

\section{Sparse recovery \cite{DBLP:journals/pieee/GilbertI10} 
A related problem to what we considered so far. Consider $A \in \mathbb{R}^{m \times n}$ for some small $m$.

\begin{definition}[Exact sparse recovery]
Given $Ax$ where $x$ is $k$-sparse ($\|x\|_0 \le k$), recover $x$.
\end{definition}

\begin{definition}[Noisy sparse recovery]
Given $Ax$, find $x'$ that is $k$-sparse such that
$$\|x - x'\| \le C \min_{z : \|z\|_0 \le k} \|x - z\|.$$
\end{definition}
(Note that we do not specify what norms to use in the definition.)
Our leverage is that \emph{we can choose} matrix $A$.

\subsection{Exact sparse recovery}
Assume the signal is non-negative, that is $x \ge 0$. Assume the simplest scenario: signal is $1$-sparse.

Then we just use the simplest matrix, with $m = \log_2 n$, and have it so $A_{i,j} = 1$ iff $i$-th bit of $j$ is $1$ and $0$ otherwise. It is easy to see that $x_j$ is ``copied'' to unique combination (depending on binary representation of $j$) of coordinates in $y = Ax$.

Example:
$$\begin{bmatrix} 0 & 0 & 0 & 1 & 1 & 1 & 1 \\ 0 & 1 & 1 & 0 & 0 & 1 & 1 \\ 1 & 0 & 1 & 0 & 1 & 0 & 1 \end{bmatrix}
$$
\textbf{Property:} In such binary matrix columns are unique, and no column is all-0.

Can we do have matrix such that it can distinguish between $1$-sparse and $2$-sparse? (Doesn't need to uniquely decode $2$-sparse, just be able to say that signal is not $1$-sparse.)

Have $m = 2 \log_2 n$ output values. We just look at non-zero values at the output, ignoring the magnitude. For each bit of position of input, we encode one output for $1$ and one for $0$ there.

$$\begin{bmatrix} 0 & 0 & 0 & 0 & 1 & 1 & 1 & 1 \\ 1 & 1 & 1 & 1 &  0 & 0 & 0 & 0 \\  0 & 0 & 1 & 1 & 0 & 0 & 1 & 1 \\ 1 & 1 & 0 & 0 & 1 & 1 & 0 & 0  \\ 0 & 1 & 0 & 1 & 0 & 1 & 0 & 1 \\ 1 & 0 & 1 & 0 & 1 & 0 & 1 & 0 \end{bmatrix}
$$
\textbf{Property:} In such matrix, no column covers any other column, and any two distinct columns have at least one intersection (common $1$).


It is actually better to talk in terms of combinatorial properties. Instead of matrices and columns, we want to talk about sets (column $\to$ set, positions of 1's $\to$ elements of set).
So in this setting, we have $n$ sets $F_1,\ldots,F_n \subseteq [m]$, and we require some combinatorial properties of sets. E.g. $\forall_{i\not=j} F_i \not\subseteq F_j$.

\begin{definition}
$k$-separable set family: for any $I_1,I_2 \subseteq [n]$, such that $|I_1| = |I_2| = k$ and $I_1 \not= I_2$, we have
$$ \bigcup_{i \in I_1} F_i \not= \bigcup_{i \in I_2} F_i$$
\end{definition}

Essentially: any $k$-sparse signal gets unique support on the image side, and can be decoded.

Issue: size of codes (value of $m$), time to decode (better than $\bigo(n^k)$, can be easily done in $\bigo(n)$, but there are ways to do it in time $\textrm{poly}(k,\log n)$).

Equivalent property: 
\begin{definition}
$k$-disjoint set family:
for any $I \subseteq [n], |I| = k$ and $t \not\in I$, there is 
$$F_t \not\subseteq \bigcup_{i \in I} F_i$$
\end{definition}

Explicit construction: \cite{doi:reed-solomon}:

consider polynomials of degree $d$, modulo $p$. Each polynomial $f(x)$ $\to$ its graph $\{(0,f(0)), (1,f(1)), \ldots, (p-1,f(p-1))\} \subseteq [p] \times [p]$. There are $p^d$ distinct polynomials. We thus create family of sets, where with each set (indexed by polynomial) we associate its graph. Thus $m = p^2$, $n = p^d$.

What property do we have? Each two sets overlap in at most $d$ values, that is for $f \not = g$ there is $|F_f \cap F_g| \le d$. Thus we need $dk + 1 \le p$.

We have $d = \frac{\log n}{\log p}$ and $d \le p/k$, so best is to set $p \approx \frac{k}{\log k} \log n$ and $d \approx \frac{\log n}{\log k}$. Then $m \approx \frac{k^2}{\log^2 k} \log^2 n$.


Also: randomized construction with $m =\bigo(k^2 \log n)$. Also, any slow-decoding can be made fast-decoding, at the cost of increasing $m$ by a factor of $\log n$.

Literature: super-imposed codes, $k$-cover-free families, nonadaptive group testing.
\subsection{Count Min as sparse recovery}
Recall count min: $t = \bigo(\varepsilon^{-1})$, $r = \bigo(\log n)$ hash functions $[n] \to [t]$.
\begin{itemize}
\item $x[i]$ is mapped to $\forall_j y[j \cdot t + h_j(i)]$ (linearly)
\item $x[i]$ value is queried from median of $\forall_j y[j \cdot t + h_j(i)]$
\end{itemize}
This of course easily can be written in the matrix form, with $m = rt = \bigo(\frac{\log n}{\varepsilon})$. 
The guarantee is that query for $x[i]$ return $x'$ such that $x' = x[i] \pm \varepsilon \|x\|_1$, whp.

\paragraph{Slow sparse recovery}
Run Count Min with $\varepsilon = \frac{1}{3k}$, and query \emph{every} $x[i]$ for $i \in [n]$. Create a list $L = \{ i : |x'[i]| \ge  2\varepsilon \cdot \|x\|_1 \}$. (This assumes knowledge of $\|x\|_1$, but this can also be recovered using \emph{linear sketching} from previous lectures.)

All heavy elements (such that $|x[i]| \ge 3 \varepsilon \|x\|_1$) are on the list, and every element in $L$ has the property that $|x[i]| \ge \varepsilon \|x\|_1$). Thus $|L| \le 1/\varepsilon = 3k$ (is sparse). We return sparse vector $v$ where $v[i] = x'[i]$ iff $i \in L$, and otherwise $v[i] = 0$. The guarantee is
$$ \|x-v\|_\infty \le 3\varepsilon \|x\|_1 = \frac{1}{k} \|x\|_1.$$


Improved analysis of Count Min: in the original analysis, we look at expected mass of elements from $x$ colliding with particular hash function, being $\sim \varepsilon \|x\|_1$. However, if we split $x$ into two vectors, $x_{(k)}$ and $x_{\text{tail}(k)}$, where first gets top-$k$ heaviest values, and second gets all the rest, we have the following:
\begin{itemize}
\item When hashing $x[i]$, it has $\bigo(1/k)$ ppb of colliding with each of elements from $x_{(k)}$ over a single hash function.
\item So with constant probability (lets say 7/8), $x[i]$ has no collision with any element from $x_{(k)}$.
\item The expected mass of collision with $x_{\text{tail}(k)}$ is $\bigo(1/k) \|x_{\text{tail}(k)}\|$.
\item So with ppb lets say 7/8, its at most $\bigo(1/k) \|x_{\text{tail}(k)}\|$ (with some larger constant than above).
\item By union bound, the error is at most $\bigo(1/k) \|x_{\text{tail}(k)}\|$ with ppb at least 3/4.
\item Taking median over $\bigo(\log n)$ hash function gives whp bound.
\end{itemize}

This gives us better guarantee:
$$ \|x-v\|_\infty \le  \frac{C}{k} \|x_{\text{tail}(k)}\|_1 = \frac{C}{k} \min_{z : \|z\|_0 \le k} \|x-z\|_1,$$
where $v$ is $3k$-sparse.


Better runtime (polylog): hierarchical structure (binary tree).


\subsection{Count-sketch as sparse recovery}
Similar analysis (with tail and heavy part of vectors) gives us
$$ \|x - v\|_\infty \le \frac{C'}{\sqrt{k}} \|x_{\text{tail}(k)}\|_2 =  \frac{C'}{\sqrt{k}} \min_{z : \|z\|_0 \le k} \|x-z\|_2$$

\bibliographystyle{alpha}
\bibliography{bib}

\end{document}


