\documentclass[11pt]{article}
\usepackage{latexsym}
\usepackage{amsmath,amssymb,amsthm}
\usepackage{epsfig}
\usepackage[right=0.8in, top=1in, bottom=1.2in, left=0.8in]{geometry}
\usepackage{setspace}
\usepackage[utf8]{inputenc}
%\spacing{1.06}

\newcommand{\handout}[5]{
  \noindent
  \begin{center}
  \framebox{
    \vbox{\vspace{0.25cm}
      \hbox to 5.78in { {University of Wrocław:\hspace{0.12cm}Algorithms for
          Big Data (Fall'19)} \hfill #2 }
      \vspace{0.48cm}
      \hbox to 5.78in { {\Large \hfill #5  \hfill} }
      \vspace{0.42cm}
      \hbox to 5.78in { {#3 \hfill #4} }\vspace{0.25cm}
    }
  }
  \end{center}
  \vspace*{4mm}
}

\newcommand{\lecture}[4]{\handout{#1}{#2}{#3}{Scribe:\hspace{0.08cm}#4}{Lecture #1}}

\newtheorem{theorem}{Theorem}
\newtheorem{corollary}[theorem]{Corollary}
\newtheorem{lemma}[theorem]{Lemma}
\newtheorem{observation}[theorem]{Observation}
\newtheorem{example}[theorem]{Example}
\newtheorem{definition}[theorem]{Definition}
\newtheorem{claim}[theorem]{Claim}
\newtheorem{fact}[theorem]{Fact}
\newtheorem{property}[theorem]{Property}
\newtheorem{assumption}[theorem]{Assumption}

\newcommand{\E}{{\mathbb E}}
\DeclareMathOperator{\var}{Var}
\newcommand{\eps}{\varepsilon}
\newcommand{\bigo}{\mathcal{O}}
\setcounter{MaxMatrixCols}{20}

\begin{document}

\lecture{14: Caching}{ 27/01/2020}{Lecturer: \emph{Przemysław Uznański }}{-}

\section{Cache-aware algorithms \cite{DBLP:conf/dagstuhl/KowarschikW02}}
DAM model:
\begin{itemize}
\item  CPU
\item cache (with fast access) of size $M$, $M/B$ blocks of size $B$
\item memory/disk (with slow access) of size $\infty$
\end{itemize} 
Cost is associated with number of memory accesses. Assume CPU cost is negligible, and actual cost comes from moving things to i from cache.

Example1: scanning $N$ consecutive memory cells takes $N/B$ memory transfers.

Example2:Accessing random $N$ memory cells takes $N$ memory transfers.

Example3: Binary search: $\log(N/B)$ (not really any significant gain)

\begin{enumerate}
\item B-trees, with branching factor $\Theta(B)$. Tree depth is $\log_B N$.
\item $B^\varepsilon$-trees: each node is a buffer of size $B$, with $B^\varepsilon$ pivots. Insert amortizes and costs $\frac{\log_B N}{\varepsilon B^{1-\varepsilon}}$, queries cost $\frac{\log_B N}{\varepsilon}$. Deletes by tombstones.
\item Sorting $\bigo(\frac{N}{B} \log_{M/B} \frac{N}{B})$ by $M/B$-way mergesort.
\end{enumerate}
\section{Cache-oblivious algorithms \cite{DBLP:conf/spaa/BenderFFFKN07}}
Desing of cache-aware algorithms requires fine-tuning to parameters of the model. In modern systems we have many levels of caching...

The cache-oblivious model: do the algorithm that works well for (almost) any setting of parameters, as algorithm does not know $B$ or $M$.
\begin{itemize}
\item Automatic block transfers triggered by word access with \emph{offline optimal block replacement}.
\item FIFO or LRU is $2$-competetive given cache of $2 \times$ size.
\item In fact it is OK to show that ANY caching strategy kind-of works.
\end{itemize}

Adapts to multi-level hierarchy.


Search trees: $\bigo(\log_B N)$. Static search tree - simulate B-tree on classic binary tree via memory placement. Take full binary tree on $N$ nodes, cut it in half (height), so top is $\sqrt{N}$ nodes (call it $T$) and bottom is $\sqrt{N}$ trees ($T_1,\ldots,T_{\sqrt{N}}$). Place in memory: place $T$, then $T_1, \ldots, T_{\sqrt{N}}$, call recursively (van Emde Boas layout).

Analysis: cut in half until height piece size $\le B$. So its also $\ge \sqrt{B}$. Height of a piece is between $\log B$ and $\frac{1}{2} \log B$. Number of pieces along path to root is $\le \frac{\log N}{\frac{1}{2} \log B}$, and each piece is on at most $2$ blocks.

COLA (Cache-Oblivious Lookahead Array)\cite{DBLP:conf/icalp/BenderCR02}:
\begin{itemize}
\item $\log N$ levels
\item $i$-th level contains $2^i$ elements, either completely full or completely empty
\item each level is sorted
\end{itemize}

Insert: $\frac{\log N}{B}$ amortized.
Naive searches: bin-search in each level, so $\log^2 N$. Refine by adding lookahead pointers: each fourth element from level $i$ is preserved in level $i+1$, with pointer. Then searching incurs $\log N$ cost.
\end{document}


