\documentclass[11pt]{article}
\usepackage{latexsym}
\usepackage{amsmath,amssymb,amsthm}
\usepackage{epsfig}
\usepackage[right=0.8in, top=1in, bottom=1.2in, left=0.8in]{geometry}
\usepackage{setspace}
\usepackage[utf8]{inputenc}
\usepackage[colorlinks=true,urlcolor=Blue,citecolor=Blue,linkcolor=BrickRed]{hyperref}
\usepackage[dvipsnames,usenames]{xcolor}
%\spacing{1.06}

\newcommand{\handout}[5]{
  \noindent
  \begin{center}
  \framebox{
    \vbox{\vspace{0.25cm}
      \hbox to 5.78in { {University of Wrocław:\hspace{0.12cm}Algorithms for
          Big Data (Fall'19)} \hfill #2 }
      \vspace{0.48cm}
      \hbox to 5.78in { {\Large \hfill #5  \hfill} }
      \vspace{0.42cm}
      \hbox to 5.78in { {#3 \hfill #4} }\vspace{0.25cm}
    }
  }
  \end{center}
  \vspace*{4mm}
}
\newcommand{\Ppb}{\mathbf{Pr}}
\newcommand{\Es}{\mathbf{E}}
\newcommand{\bigo}{\mathcal{O}}
\newcommand{\Var}{\mathbf{Var}}
\newcommand{\sF}{\mathcal{F}}
\newcommand{\Nat}{\mathbf{N}}
\newcommand{\Real}{\mathbf{R}}

\newcommand{\lecture}[4]{\handout{#1}{#2}{#3}{Scribes:\hspace{0.08cm}#4}{Lecture #1}}

\newtheorem{theorem}{Theorem}
\newtheorem{corollary}[theorem]{Corollary}
\newtheorem{lemma}[theorem]{Lemma}
\newtheorem{observation}[theorem]{Observation}
\newtheorem{example}[theorem]{Example}
\newtheorem{definition}[theorem]{Definition}
\newtheorem{claim}[theorem]{Claim}
\newtheorem{fact}[theorem]{Fact}
\newtheorem{assumption}[theorem]{Assumption}
\newtheorem{remark}[theorem]{Remark}


\begin{document}

\lecture{1: Approximate Counting, Distinct Elements}{ 07/10/2019}{Lecturer: \emph{Przemysław Uznański }}{\emph{Mikołaj Słupiński}}


\section{Introduction}
\subsection{Topics during the course}
\begin{itemize}
    \item Streaming (counting, heavy hitters, norm estimation, sampling)
 ($\sim4$ Lectures)
    \item Dimensionality reduction and sparse linear algebra (e.g. JL, approx matrix mul, compressed sensing)
 ($\sim4$ Lectures)
    \item Applications (geometry algo, coresets, graph algorithms, ANN, sliding window) ($\sim4$ Lectures)
\end{itemize}
\subsection{Motivation}
Linear time/space algorithms are not good enough with modern datasets and their volume. Typical problem we are dealing with in this course: here is a stream of data, process it in a small space to compute output X. Usually there is a lower-bound preventing us to do it in a very small space \emph{exactly}. Hence we need to relax our problem to achieve very efficient (in space and time) algorithms.
Examples:
\begin{itemize}
    \item Think of any recommendation system, where each user has assigned highly dimensional vector of preferences. We want to test similarity/dissimilarity of user profiles.
    \item Database with approximate index (Approx Membership Queries), to quickly eliminate queries for elements that are not in the DB, except for few false positives.
    \item Lossy compression of audio or images selects heavy hitters in the frequency domain. How to find them without computing FFT explicitly?
    \item Count distinct elements in a stream, or maintain statistics in a continuous stream of updates (router + number of unique IP).

\end{itemize}
\subsection{Techniques}
\begin{itemize}
    \item Probabilistic tools  - few probabilistic bounds are good enough 90\% of the time, sometimes we will need to go a little bit deeper (fancy distributions),
    \item relaxing problem: $1\pm \varepsilon$ approximation and $1-\delta$ correctness guarantee,
    \item linear algebra,
    \item trace amounts of combinatorics and ``typical'' A\&DS - that's why it might be tricky for CS students.
\end{itemize}
\section{Approximate counting} 
The problem is to maintain a counter that supports following operations:
\begin{align*}
\text{reset(),  } &[n\leftarrow 0]\\
\text{inc(),    } &[n\leftarrow n+1]\\
\text{query(),  } &[\text{output n}]
\end{align*}

Simple lower bound of $\log(n)$ bits for exact (information-theoretic lower bound).

\paragraph{Goal:} algorithm that queried outputs $n’$ such that $\Ppb( |n-n'| > \varepsilon n ) < \delta$.
\subsection{Morris’ algorithm \cite{DBLP:journals/cacm/Morris78a}}
\paragraph{Local state:} X [int], represents $n \sim 2^X$. The crucial part of algorithm is to design how we increase $X$.

\paragraph{Inc: }$X \leftarrow X+1$ with some small probability ($\sim 2^{-X}$), with intuition being that the ppb of $n$ being exactly $2^{X+1}-1$ is $2^{-X}$.

Let us analyze increment probability = $2^{-X}$. Let $X_n$ be random variable denoting state of algorithm after $n$ increases.
\begin{theorem}
\begin{align}
    \Es[2^{X_0}] &= 2^{X_0} = 1 \\
    \Es[2^{X_n}] &= n+1 \text{  by induction}
\end{align}
\end{theorem}
\begin{proof}
\begin{align*} \Es 2^{X_{n+1}} &=\sum_{j=0}^{\infty} \Ppb \left(X_{n}=j\right) \cdot \Es \left(2^{X_{n+1}} | X_{n}=j\right) \\ &=\sum_{j=0}^{\infty} \Ppb \left(X_{n}=j\right) \cdot\left(2^{j}\left(1-\frac{1}{2^{j}}\right)+\frac{1}{2^{j}} \cdot 2^{j+1}\right) \\ &=\sum_{j=0}^{\infty} \Ppb \left(X_{n}=j\right) 2^{j}+\sum_{j} \Ppb \left(X_{n}=j\right) \\ &=\Es 2^{X_{n}}+1 \\ &=(n+1)+1 \end{align*}
\end{proof}

Morris algorithm output: $Z = 2^{X_{n}}-1   \leftarrow$, which is an unbiased estimator of $n$ (that is $\Es[Z] = n$).

\subsubsection{Analysis of variance to extract guarantees:}

\begin{theorem}
We show inductively that $\Es[2^{2X_n}] = 3/2 n^2 + 3/2 n + 1$.
\end{theorem}
\begin{proof}
see exercise
\end{proof}

Since
\begin{align*}
    \Var[Z] &= \Var[2^{X_n}] \\
    &= \Es[2^{2 X_n} ] - (\Es[2^{X_n}])^2 \\
    &= \frac{3}{2} n^2 + 3/2 n + 1 - (n+1)^2 \\
    &= \frac{1}{2} n^2 - \frac{1}{2}n,
\end{align*}
by Chebyshev’s inequality $\Ppb( |Z-n| > \varepsilon n) \leq 1/(2 \varepsilon^2)$.

This only gives failure probability $\delta < \frac{1}{2}$ for $\varepsilon > 1$, which is not very informative: (large) constant approximation with constant probability. But that was to be expected: our algorithm only outputs powers of two, so it cannot do much better job.
\subsection{Morris+}
Repeat $k$ times independently, take average of estimations.
Since variance is additive: $\Var(Z') = \frac{1}{k^2}  (\Var(Z_1) + \Var(Z_2) + \dots + \Var(Z_k)) = 1/k  \Var(Z)$ so number of iterations necessary becomes: $k = \bigo(\frac{1}{\varepsilon^2 \delta})$
(ok for 9/10 ppb of correctness, bad for whp correctness).

\subsection{Morris++}
Run $t$ copies of Morris+ algorithm, each with $\delta = \frac{1}{3}$ and take median of estimations as a final estimation. Each estimation is ok with probability $\geq \frac{2}{3}$, so for the median to fail at least $\frac{1}{6}$ fraction of estimations need to fail (all too large or all too small)
Chernoff bound gives us:
\begin{equation}
\Ppb \left(\sum_{i=1}^{t} Y_{i} \leq \frac{t}{2}\right) \leq \Ppb \left(\left|\sum_{i=1}^{t} Y_{i}-\Es \sum_{i=1}^{t} Y_{i}\right| \geq \frac{t}{6}\right) \leq 2 e^{-t / 3}<\delta
\end{equation}
for $t = \Theta(\lg (1 / \delta))$.
Final \textbf{bit} complexity $\bigo(\log \log (n/(\varepsilon \delta)) \frac{1}{\varepsilon^2} \log(\frac{1}{\delta}))$.

\paragraph{Lower bound:} $\Omega(\log \log_{1+\varepsilon} n) = \Omega(\log(1/\varepsilon) + \log \log n)$ (for $\delta=0$, its trickier to prove lowerbound involving $\delta$)


\section{Distinct elements}

\paragraph{Input:} Stream of values $i_1, i_2, …, i_m$ from $[n]$
query() $\leftarrow$ number of distinct elements

\paragraph{Trivial solution:} remember the stream, bitvector
\subsection{Flajolet Martin \cite{DBLP:journals/jcss/FlajoletM85}}
Pick a hash function $h: [n] \rightarrow [0,1]$     (for a moment let us assume ideal real numbers, and perfectly random hash function).

\begin{enumerate}
\item initially $Z=1$
\item input $X$: $Z = \min(Z, h(X))$
\item estimator: $Y = 1/Z - 1$
\end{enumerate}

\begin{observation}
Repeats do not affect Z.
\end{observation}

If $t$ is the number of distinct elements, then $Z = \min(r_1, r_2, \dots, r_t)$ where $r_i$ are all independent and from $[0,1]$.

\begin{lemma}
\begin{equation}
    \Es[Z] = \frac{1}{t+1}
\end{equation}
\end{lemma}
\begin{proof}
Pick fresh $A$ at random from $[0,1]$. By symmetry, 
$$\Es[Z] = \Ppb[A<Z] = \Ppb[A\text{ is minimal among }{A,r_1,\dots,r_t}] = \frac{1}{(t+1)}.$$
\end{proof}
\begin{lemma}
\begin{equation}
\Es[Z^2] \leq \frac{2}{(t+1)(t+2)}
\end{equation}
\end{lemma}
\begin{proof}
Pick fresh $A, B$ at random from $[0, 1]$. By symmetry, $\Es[Z^2] = \Ppb[A<Z \wedge B<Z] = \frac{2}{(t+1)(t+2)}$ 
\end{proof}
\begin{proof}[Alternative proof]
\begin{align*}
\Es\left[Z^{2}\right] &=\int_{0}^{\infty} \Ppb\left(Z^{2}>\lambda\right) d \lambda \\
&=\int_{0}^{\infty} \Ppb(Z>\sqrt{\lambda}) d \lambda \\
&=\int_{0}^{1}(1-\sqrt{\lambda})^{t} d \lambda \\
&=2 \int_{0}^{1} u^{t}(1-u) d u \quad[u=1-\sqrt{\lambda}] \quad=\frac{2}{(t+1)(t+2)} \end{align*}
\end{proof}
\begin{equation}
\Var[Z]=\frac{2}{(t+1)(t+2)}-\frac{1}{(t+1)^{2}}=\frac{t}{(t+1)^{2}(t+2)}<(\Es[Z])^{2}
\end{equation}

\begin{remark}
Applying Chebyshev's inequality $\rightarrow$ results in a guarantee of a (large) constant approximation with lets say $\frac{9}{10}$ probability.
\end{remark}

\paragraph{Issue:} $\Es[\frac{1}{Z}] \neq \frac{1}{\Es[Z]}$, but concentrating $Z$ with $1+\varepsilon$ multiplicative error will give $1+\varepsilon$ multiplicative error for $\frac{1}{Z}$.

\subsection{FM+}
To reach better approximation guarantee, we need to concentrate our output around expected value.
\paragraph{Approach 1} copy approach from Morris’ algorithm - ``repeat k times and take average''
to improve variance, set $k = \bigo(\frac{1}{\varepsilon^2})$ for $\frac{9}{10}$ probability of $1+ \varepsilon$ approximation.

\paragraph{Approach 2} replace ``take minimum'' with ``take k-th smallest value''
(to be analyzed $\rightarrow$ exercise).

\subsection{FM++}
To improve probability of success, repeat FM+ algorithm $t = \bigo(\log \delta^{-1})$ times, and take median of answers. This boosts probability of success to $1-\delta$.

Total memory complexity is

$\bigo(\log n  \frac{1}{\varepsilon^2} \log \delta^{-1})$ of \textbf{words} (each word is $\log n$ bits).

\subsection{Issues}

\paragraph{Recall} ``for a moment let us assume ideal real numbers''.

We only care about relative order of hashes, and use actual value as an estimator. Using hash-functions of form $h: [n] \rightarrow \{\frac{0}{M}, \frac{1}{M}, \dots, \frac{M-1}{M}, \frac{M}{M}\}$ for some $M = n^3$, as it only introduces small relative error (whp each hash is $\geq \frac{1}{n}$ thus relative error introduced is at most $(1+\frac{1}{n})$, and wlog $\varepsilon>\frac{1}{n}$), and whp there are no collisions of hashes.

\paragraph{Recall} ``and perfectly random hash function''.

Randomness vs. pseudorandomness $\rightarrow$ c.f. exercises

\section{Further reading}
\begin{itemize}
    \item hyperloglog algorithm, which very efficient in theory and practice, but has extremely nontrivial analysis \cite{DBLP:conf/esa/DurandF03} \cite{DBLP:conf/edbt/HeuleNH13}
    \item\ \cite{DBLP:conf/soda/Blasiok18} - optimal $\Theta(\log{n} + \frac{\log{\delta^{-1}}}{\varepsilon^2})$ bits.

\end{itemize}

\bibliographystyle{alpha}
\bibliography{bib}


\appendix
\section{Probability recap}
\begin{definition}
\begin{enumerate}
    \item The empty set is an event, $\varnothing \in \sF$
    \item Given a countable set of events $A_1, A_2, \dots$, its union is also an event, $\bigcup_{i\in\Nat}A_i \in \sF$
    \item if $A$ is an event, then so is the complementary set $A^c$
\end{enumerate}
\end{definition}

\begin{definition}
\begin{enumerate}
    \item $\Ppb(\emptyset)=0, \Ppb(\Omega)=1$
    \item if $A_1, A_2, \dots$ are mutually excluding events, then $\Ppb\left(\cup_{i=1}^{\infty} A_{i}\right)=\sum_{i=1}^{\infty} \Ppb\left(A_{i}\right)$
\end{enumerate}
A $\Ppb: \mathcal{F} \mapsto[0,1]$ satisfying these is called a probability.

The triple $(\Omega, \mathcal{F}, \Ppb)$ is called a probability space.
\end{definition}

\begin{definition}
We define conditional probability as 
\begin{equation*}
    \Ppb(A | B)=\frac{\Ppb(A \cap B)}{\Ppb(B)}
\end{equation*}
\end{definition}
\begin{theorem}
Let $B_{1}, \dots, B_{n}$ be a partition of $\Omega$, then
\begin{equation}
\Ppb(A)=\sum_{i=1}^{n} \Ppb\left(A | B_{i}\right) \Ppb\left(B_{i}\right)
\end{equation}
\end{theorem}
\begin{definition}
Events A and B are called independent if
\begin{equation}
\Ppb(A \cap B)=\Ppb(A) \Ppb(B).
\end{equation}
When$ 0 < \Ppb(B) < 1$, this is the same as
\begin{equation}
\Ppb(A | B)=\Ppb(A)=\Ppb\left(A | B^{c}\right)
\end{equation}
A family $\{A_i : i \in I\}$ of events is called independent if
\begin{equation}
\Ppb\left(\cap_{i \in J} A_{i}\right)=\prod_{i \in J} \Ppb\left(A_{i}\right)
\end{equation}
for any finite subset J of I.
\end{definition}

\begin{definition}
A random variable is
Informally: A quantity which is assigned by a random experiment.
Formally: A mapping $X : \Omega \rightarrow \Real$.
\end{definition}
\begin{definition}
The cumulated distribution function(cdf) is:
\begin{equation}
F(x)=\Ppb(X \leq x)
\end{equation}
If satisfies following properties:
\begin{enumerate}
    \item $\lim _{x \rightarrow-\infty} F(x)=0, \lim _{x \rightarrow+\infty} F(x)=1$
    \item $x<y \Rightarrow F(x) \leq F(y)$
    \item $F$ is right-continuous, ie. $F(x+h) \rightarrow F(x)$ as $h \downarrow 0$
\end{enumerate}
\end{definition}
\begin{definition}
The mean of a stochastic variable is
$$
\Es X=\sum_{i \in \mathbb{Z}} i \Ppb(X=i)
$$
in the discrete case, and
$$
\Es X=\int_{-\infty}^{+\infty} f(x) d x
$$
in the continuous case. In both cases we assume that the
sum/integral exists absolutely.
The variance of $X$ is
$$
\Var X=\Es(X-\Es x)^{2}=\Es X^{2}-(\Es X)^{2}
$$
\end{definition}

\begin{definition}
The conditional expectation is the mean in the conditional distribution
\begin{equation}
\Es(Y | X=x)=\sum_{y} y f_{Y | X}(y | x)
\end{equation}
It can be seen as a stochastic variable: Let $\psi(x)=\Es(Y | X=x)$,
then $\psi(X)$ is the conditional expectation of $Y$ given $X$
\begin{equation}
\psi(X)=\Es(Y | X)
\end{equation}
We have
\begin{equation}
\Es(\Es(Y | X))=\Es Y
\end{equation}
\end{definition}

\begin{definition}
Conditional variance $\Var(Y|X)$ is the variance in the conditional distribution.
\begin{equation}
\Var(Y | X=x)=\sum_{y}(y-\psi(x))^{2} f_{Y | X}(y | x)
\end{equation}

This can also be written as
$$
\Var(Y | X)=\Es\left(Y^{2} | X\right)-(\Es(Y | X))^{2}
$$
and can be manipulated into
$$
\Var=\Es \Var(Y | X)+\Var \Es(Y | X)
$$
which partitions the variance of $Y$.
\end{definition}

\begin{theorem}[Markov's inequality]
Let $X \geq 0$ be a random variable. Then for any $k \geq 1:$
\begin{equation}
\Ppb(X \geq k \cdot \Es[X]) \leq \frac{1}{k}
\end{equation}
\end{theorem}
\begin{theorem}[Chebyshev's inequality]
Let $X$ be a random variable. For any $k>0$:
\begin{equation}
\Ppb(|X-\Es[X]| \geq k \cdot \sqrt{\Var[X]}) \leq \frac{1}{k^{2}}
\end{equation}
\end{theorem}
\begin{theorem}[]Hoeffding bound]
Let $X_{1}, X_{2}, \ldots, X_{n} \in\{0,1\}$ be fully independent ran-dom variables. Let $X=\sum_{i} X_{i} .$ Then:
\begin{equation}
    \Ppb(|X-\Es[X]| \geq t) \leq 2 \exp \left(-\frac{t^{2}}{n}\right)
\end{equation}
\end{theorem}
\end{document}
